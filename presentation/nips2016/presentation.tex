\documentclass{beamer}
\usecolortheme{seahorse} % change this
\usepackage{graphicx}
\usepackage{inconsolata}
\usepackage{anyfontsize}




\title 
{A constructive approach for graph concepts
with long range dependencies}

\author % NEEDS MOAR INFO ,, contact etc
{Stefan Mautner }

\date 
{Bioinformatics, Freiburg University, 2017-12-09}



\begin{document}
\frame{\titlepage}



% design goals
\begin{frame}
\frametitle{What are the design goals?}

    We construct graphs; how do we define the task exactly?
    ~\\
    ~\\
    ~\\
    \emph{the constructive learning problem for finite samples} 
    ~\\
    \begin{itemize}
        \item optimize for two goals:
        \item similar probability density to train set (compare trained estimators)
        \item generated instances should  differ from originals
    \end{itemize}
\end{frame}


% so far we have the basic graphlearn
\begin{frame}
    \frametitle{Basic algorithm}
    MCMC (Markov Chain Monte Carlo)
    \begin{enumerate}
        \item start with training graph
        \item alter start graph
        \item decide weater or not to accept proposed change $\rightarrow$ goto 2
    \end{enumerate}
    Graph Grammar to increase quality of proposals
    \begin{itemize}
        \item extract graph fragments (CIPS) from training set;
            use them to alter graph as follows:
    \end{itemize}
    %IMAGE OF SOMETHING  
    \begin{figure}[ht]
        \centering
        \includegraphics[width=0.33\textwidth]{images/CIP_replacement.png}
    \end{figure}
    \small{A replacement can take place if CIPs are \emph{congruent}  }
\end{frame}





% problem with RNA 
\begin{frame}
    \frametitle{The Problem}
    when graphs become larger..
    \begin{itemize}
        \item You want to operate on a higher level 
        \item Local constraints as introduced by grammar may not be enough
    \end{itemize}
   \begin{figure}[h!]
        \centering
        \includegraphics[width=0.33\textwidth]{images/longrangedep.png}
    \end{figure}
    
    \begin{itemize}
        \item (green) use whole ``stem'' as basis for operations
        \item (blue+green) example for a long range dependency. The spheres
            might depend on the presence of one another.
    \end{itemize}
\end{frame}



% solution
\begin{frame}
    \frametitle{Solution}
   \begin{figure}[ht]
        \centering
        \includegraphics[width=0.50\textwidth]{images/nucip.png}
    \end{figure}
    \begin{itemize}
        \item Find a way to abstract your graph (right)
        \item Redefine Core as nodes from the base graph, induced by the 
            abstract graph (dark green)
        \item Redefine Interface-graph as nodes around the core,
            congruency check for CIPs includes the abstract interface
   \end{itemize}
   CIPs have a flexible shape now and include information about its surroundings
   from the abstract graph
\end{frame}





% BENCHMARK I 
\begin{frame}
    \frametitle{Evaluation I}
    \begin{itemize}
        \item We used this method to generate RNA sequences
        \item Rfam/Infernal offers a domainspecific way to evaluate sequences
        \item Also trained an Infernal model under similar conditions
    \end{itemize}

   \begin{figure}[ht]
        \centering
        \includegraphics[width=0.60\textwidth]{images/infernal_abstr.png}
    \end{figure}
\end{frame}

% BENCHMARK II
\begin{frame}
    \frametitle{Evaluation II}
    
    \begin{itemize}
        \item Similarity of the generated set vs the train set
        \item Kullback Leibler divergence to meassure the difference in probability distribution
    \end{itemize}
   \begin{figure}[ht]
        \centering
        \includegraphics[width=0.6\textwidth]{images/learningcurve.png}
    \end{figure}
    \tiny{KL Divergence:  $D_{\mathrm{KL}}(P\|Q) = \sum_i P(i) \, \log\frac{P(i)}{Q(i)}$ }
    % IMAGE OF other thing
\end{frame}

% OWARI DA 
\begin{frame}
    \frametitle{Conclustion}
    %\begin{itemize}
        Use the information of abstractions
        to increase the power of your graph grammar!
    %\end{itemize}
\end{frame}


\end{document}
